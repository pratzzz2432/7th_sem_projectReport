\chapter{LITERATURE SURVEY}
\label{chap:literature_survey}

The integration of blockchain technology into healthcare has emerged as a significant area of research, driven by the persistent challenges of data security, interoperability, and patient privacy in traditional Electronic Health Record (EHR) systems. A substantial body of literature explores the potential of blockchain to address these issues, offering a decentralized, transparent, and immutable framework for managing sensitive medical data. This chapter provides a comprehensive review of the existing research, highlighting key developments, identifying gaps, and positioning MediVault within the broader academic and technological landscape.

\section{The Case for Decentralization in Healthcare}
Traditional healthcare data management systems are predominantly centralized, relying on institutional databases that are vulnerable to single points of failure and targeted cyberattacks. The consequences of such vulnerabilities are severe, ranging from the exposure of sensitive patient information to the disruption of critical healthcare services. The literature emphasizes that centralization also creates data silos, where medical records are fragmented across various providers, leading to incomplete patient histories and hindering coordinated care \autocite{abdel2023blockchain}.

Blockchain technology offers a paradigm shift from this centralized model. By distributing data across a peer-to-peer network, it eliminates the reliance on a single trusted intermediary, thereby enhancing security and resilience. Each transaction is cryptographically linked to the previous one, creating an immutable chain of records that is transparent and auditable. This inherent trust mechanism is particularly valuable in a multi-stakeholder environment like healthcare, where patients, providers, and researchers need to interact with a shared source of truth \autocite{damar2025blockchain}.

\section{Pioneering Blockchain Solutions in Healthcare}
The exploration of blockchain in healthcare began with foundational projects that demonstrated the viability of decentralized record-keeping. One of the earliest and most cited examples is \textbf{MedRec}, developed by researchers at MIT in 2016. MedRec utilized a permissioned blockchain to manage access to patient records, providing an immutable log of all data interactions. While groundbreaking, MedRec's scope was primarily focused on record tracking and did not address the complexities of data anonymization or patient consent for research \autocite{ekblaw2016case}.

Building on this, subsequent projects sought to enhance interoperability and data sharing. \textbf{FHIRChain} emerged as a notable effort to combine the Fast Health Interoperability Resources (FHIR) standard with blockchain. The goal was to create a standardized, decentralized framework for exchanging clinical data. However, a key limitation of FHIRChain was its reliance on off-chain storage for the actual medical data, which meant that the data itself was still susceptible to the vulnerabilities of centralized databases \autocite{zhang2018fhirchain}.

\textbf{HealthChain} and \textbf{BlockHealth} further advanced the field by exploring decentralized access control and tokenized data sharing, respectively. HealthChain used Ethereum smart contracts to manage permissions, but it lacked a robust mechanism for patient-driven consent. BlockHealth introduced the concept of a token-based economy for health data, but its framework for ethical research participation remained underdeveloped \autocite{hylock2019blockchain, griggs2018blockhealth}.

\section{The Critical Role of Patient Consent and Data Anonymization}
While early blockchain projects focused on the technical aspects of data security and interoperability, more recent research has shifted towards the ethical and patient-centric dimensions of data sharing. The literature highlights a critical gap in existing systems: the lack of a transparent and dynamic mechanism for managing patient consent, particularly for the secondary use of data in research. Traditional consent models are often static and buried in lengthy terms and conditions, giving patients little control over how their data is used \autocite{abdel2023blockchain}.

Blockchain offers a transformative solution to this challenge. By recording consent on an immutable ledger, it provides a transparent and auditable trail of all permissions granted and revoked. This empowers patients to become active participants in the data economy, with the ability to control who can access their data and for what purpose.

\section{Positioning MediVault in the Research Landscape}
The development of MediVault is directly informed by the insights and limitations identified in the existing body of literature. While previous projects have made significant strides in demonstrating the technical feasibility of blockchain in healthcare, they have often overlooked the critical importance of a patient-centric approach to consent and research participation.

MediVault distinguishes itself by introducing a dedicated \textbf{Research Consent Layer}, which is built directly into the core architecture of the system. This layer provides a granular and dynamic consent mechanism that is both user-friendly and cryptographically secure. Patients can grant or revoke consent for their anonymized data to be used in research at any time, with their decisions recorded immutably on the blockchain.

This approach addresses a key ethical and practical challenge in medical research. By providing a secure and transparent platform for data sharing, MediVault aims to accelerate scientific discovery while upholding the highest standards of patient privacy and autonomy. The integration of a dedicated consent layer represents a significant contribution to the field, moving beyond the purely technical applications of blockchain to address the more nuanced challenges of ethical data stewardship.

\section{The Future of Blockchain in Healthcare}
The literature suggests that the integration of blockchain into healthcare is still in its early stages, with significant potential for future innovation. As the technology matures, it is expected to play a crucial role in enabling a more patient-centric, secure, and interoperable healthcare ecosystem. Future research will likely focus on the scalability of blockchain networks, the integration of artificial intelligence for data analysis, and the development of standardized protocols for cross-institutional data sharing \autocite{damar2025blockchain}.

MediVault is positioned at the forefront of this evolving landscape, offering a practical and ethically grounded solution to some of the most pressing challenges in healthcare data management. By prioritizing patient consent and data anonymization, it provides a robust framework for the responsible and effective use of medical data in the digital age.

\section{Comparative Analysis of Blockchain Projects}
To provide a clear overview of how MediVault compares to the pioneering projects discussed in this chapter, the following table summarizes their key features and limitations.

\begin{table}[h!]
\centering
\caption{Comparative Analysis of Blockchain-Based Healthcare Projects}
\label{tab:comparison}
\begin{tabular}{|l|l|l|l|}
\hline
\textbf{Project} & \textbf{Key Features} & \textbf{Limitations} & \textbf{MediVault's Contribution} \\ \hline
\textbf{MedRec} & \begin{tabular}[c]{@{}l@{}}- Decentralized record tracking\\ - Immutable log of data access\end{tabular} & \begin{tabular}[c]{@{}l@{}}- Limited to record tracking\\ - No data anonymization\\ - Basic consent model\end{tabular} & \begin{tabular}[c]{@{}l@{}}Advanced, granular consent\\ for research\end{tabular} \\ \hline
\textbf{FHIRChain} & \begin{tabular}[c]{@{}l@{}}- Promotes interoperability\\ - Uses FHIR standard\end{tabular} & \begin{tabular}[c]{@{}l@{}}- Relies on off-chain storage\\ - Centralized data vulnerability\end{tabular} & \begin{tabular}[c]{@{}l@{}}Fully decentralized architecture\\ for data and logic\end{tabular} \\ \hline
\textbf{HealthChain} & \begin{tabular}[c]{@{}l@{}}- Decentralized access control\\ - Uses Ethereum smart contracts\end{tabular} & - Lacks patient-driven consent & \begin{tabular}[c]{@{}l@{}}Patient-centric consent model\\ with revocable permissions\end{tabular} \\ \hline
\textbf{BlockHealth} & - Tokenized data sharing & \begin{tabular}[c]{@{}l@{}}- Underdeveloped research\\ participation framework\end{tabular} & \begin{tabular}[c]{@{}l@{}}Dedicated Research Consent\\ Layer for ethical data sharing\end{tabular} \\ \hline
\end{tabular}
\end{table}

This comparative analysis highlights the unique contribution of MediVault, particularly in its emphasis on a dedicated Research Consent Layer. By addressing the limitations of previous projects, MediVault offers a more comprehensive and patient-centric solution for blockchain-based healthcare data management.
