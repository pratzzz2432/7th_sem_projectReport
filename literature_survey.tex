\chapter{LITERATURE SURVEY}
\label{chap:literature_survey}

The concept of integrating blockchain with healthcare data systems has been explored in several research studies, each contributing to the field while also revealing gaps that MediVault aims to address. Early pioneering work such as \textbf{MedRec (MIT, 2016)} demonstrated the feasibility of using blockchain for healthcare records, but its functionality was limited to simple record tracking and it lacked crucial data anonymization capabilities.

Subsequent developments sought to improve upon this foundation. \textbf{FHIRChain (2018)} aimed to enhance interoperability between medical institutions by leveraging blockchain, yet it continued to rely on centralized databases for the actual storage of data, thus retaining a key vulnerability. Similarly, \textbf{HealthChain (2019)} utilized Ethereum for decentralized access control but did not provide a mechanism for patient-driven consent, leaving a critical aspect of patient autonomy unaddressed. More recently, \textbf{BlockHealth (2020)} explored the concept of tokenized data sharing but did not incorporate a framework for ethical research participation.

These works collectively highlight a significant trend: while blockchain can effectively ensure immutability and transparency in transaction logging, the nuanced requirements of patient consent and privacy-preserving research have remained largely unaddressed. MediVault distinguishes itself from these predecessors by introducing a dedicated \textbf{Research Consent Layer}. Through this layer, each patient can explicitly decide whether their anonymized medical data can be used for research purposes. Unlike conventional systems where consent is often buried in terms and conditions, this consent is stored and tracked immutably on the blockchain. This ensures that researchers can only access data from patients who have actively opted in, and more importantly, that this participation can be revoked by the patient at any time with immediate effect.
