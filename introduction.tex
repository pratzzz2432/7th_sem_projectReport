\chapter{INTRODUCTION}
\label{chap:introduction}

\section{Outline}
In an era where healthcare is becoming increasingly digital, the volume of patient data being generated each day has grown exponentially. However, this growth has not been matched by equivalent progress in data ownership, privacy, or interoperability. Medical records are typically stored within centralized databases managed by hospitals, clinics, and laboratories — systems that are often incompatible with each other and vulnerable to breaches. Patients rarely have visibility or control over their personal health information, and data-sharing between institutions remains inefficient and insecure.

To address these critical issues, MediVault has been conceptualized and developed as a blockchain-based healthcare data sharing system. It enables patients to securely store, manage, and share their encrypted medical data through a decentralized and transparent network. Beyond security, MediVault extends into the ethical realm by introducing an Anonymized Opt-in Research Module. This feature allows patients to voluntarily contribute de-identified medical data to research projects while maintaining full control over their consent — achieving a delicate balance between individual privacy and collective scientific progress. Built using Ethereum blockchain, Solidity smart contracts, Truffle, Ganache, Web3.js, and React.js, MediVault creates a fully decentralized ecosystem where every operation — from record creation to data sharing — is verified, recorded, and immutable.

\section{Problem Statement}
Traditional Electronic Health Record (EHR) systems rely on centralized architectures that introduce several issues. These include security vulnerabilities, unauthorized data access, duplication of medical records, and lack of transparency in data handling. Moreover, in many research scenarios, hospitals anonymize and sell patient data without explicit consent — raising ethical and legal concerns. MediVault addresses these problems by establishing a patient-centric data ownership model on the blockchain. Every piece of health information and every permission is governed by smart contracts, ensuring that data cannot be altered, deleted, or shared without patient approval. The system thereby restores autonomy to the patient while preserving data security and accountability.

\section{Motivation}
The motivation behind MediVault arises from two key needs: patient empowerment — giving individuals direct control over their medical information — and ethical research enablement, which allows large-scale health data analysis while respecting personal privacy. The growing misuse of medical data, coupled with the necessity of data-driven healthcare innovation, demands a system that protects both privacy and progress. MediVault provides precisely this — a platform where patients become active participants, choosing when and how their data can be shared for research, and ensuring transparency through blockchain verification.

\section{Purpose}
The purpose of MediVault is to design and develop a secure, decentralized healthcare data sharing platform that enables patient ownership and management of medical records, secure data exchange between patients, doctors, and researchers, end-to-end encryption and tamper-proof data storage using blockchain, and voluntary and revocable consent for anonymized research participation. By achieving these goals, MediVault envisions a healthcare ecosystem that is secure, ethical, transparent, and fundamentally patient-first.
