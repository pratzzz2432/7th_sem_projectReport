\chapter{RESULTS}
\label{chap:results}

The MediVault system performed successfully in all functional areas during testing, validating the effectiveness of its decentralized design. Patients were able to seamlessly register using their MetaMask wallets, add encrypted medical data to their profiles, and selectively grant access to doctors. The access control mechanism proved to be robust, with all permission changes being accurately reflected on the blockchain.

A key success was the performance of the anonymized research module. When this feature was activated by a patient, their non-identifiable data became available for aggregation. Researchers, using their designated interface, could access these de-identified datasets, which were derived exclusively from patients who had explicitly opted in. This demonstrated a practical and ethical way to facilitate large-scale data analysis while preserving patient privacy.

Each operation was transparently recorded on the blockchain, providing a permanent and immutable audit trail. The Ganache GUI was instrumental in verifying this, as it displayed every transaction related to data uploads, access grants, and consent changes, confirming that all activities were traceable and tamper-proof. The average transaction confirmation time on the local Ethereum network was consistently between 4 and 6 seconds. Gas consumption for smart contract interactions was successfully optimized to remain below 500,000 units, indicating efficient contract design. The user interface remained responsive throughout testing, with MetaMask providing seamless and intuitive transaction confirmation prompts, resulting in a positive user experience. The anonymized research feature proved to be particularly powerful, effectively allowing for valuable data aggregation without ever exposing personal identifiers.
